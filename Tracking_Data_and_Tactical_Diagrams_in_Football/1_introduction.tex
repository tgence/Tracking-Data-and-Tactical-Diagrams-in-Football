% Some commands used in this file
\newcommand{\package}{\emph}

\chapter{Introduction}
\label{chap:1_introduction}
Tactics boards have long been essential tools for visualizing and communicating football strategies at all levels of the game. Traditional magnetic whiteboards and manual sketches, while accessible and popular, are fundamentally limited: they cannot store scenarios, replay actions, scale representations, or accurately model complex tactical relationships such as formations and coordinated movements~\cite{morgulev2018sports, vidal2022automatic}.

In response to these limitations, digital tactical boards have emerged, ranging from commercial products~\cite{tacticalpad2025, easy2coach2025, tacticalboard2025} to research prototypes~\cite{liu2024smartboard, fernandez2020soccermap}. Digital platforms now offer features such as interactive scenario editing, session storage, and, in some cases, the integration of spatio-temporal match data~\cite{bassek2025integrated, wu2018forvizor}. However, most commercial tools remain focused on manual drawing and general usability, often lacking analytical or data-driven capabilities and rarely supporting advanced football-specific structures~\cite{tacticalista2025}. In contrast, academic systems tend to prioritize novel visualizations and analytics, but frequently overlook usability, workflow integration, and requirements for widespread practical adoption~\cite{delibas2019interactive, seebacher2021investigating}.

Another persistent challenge is the lack of standardization for visual conventions, symbology, or interaction models in commercial and academic tools~\cite{walsh2023lack, bertin2011semiology}. This fragmentation creates inconsistencies in tactical communication, hinders comparison between approaches, and complicates the integration of advanced analysis or visualization techniques~\cite{perin2013soccerstories, sarmento2014match}.

This thesis addresses these gaps on two fronts. First, it surveys and critiques the current ecosystem of digital tactic boards, identifying key user needs and open problems, particularly around multitouch interaction, domain-specific representations, and exploratory workflows. Second, it presents the design and implementation of a fully functional prototype, highlighting both implemented features and those identified as essential but missing from the current landscape. The outcome is both a demonstration of current possibilities and a set of practical design guidelines to inform the next generation of digital tactical board software for football.